\documentclass[10pt,danish,t,10pt]{beamer}
\usepackage{lmodern}
\usepackage[T1]{fontenc}
\usepackage[utf8]{inputenc}
\setlength{\parskip}{\smallskipamount}
\setlength{\parindent}{0pt}
\usepackage{babel}
\usepackage{amsmath}
\usepackage{amssymb}
\usepackage{graphicx}
\usepackage{comment} % For longer notes
\ifx\hypersetup\undefined
  \AtBeginDocument{%
    \hypersetup{unicode=true}
  }
\else
  \hypersetup{unicode=true}
\fi
%\usepackage{breakurl}

\usepackage{tikzsymbols}% Smileys and stuf

\makeatletter
%%%%%%%%%%%%%%%%%%%%%%%%%%%%%% Textclass specific LaTeX commands.
% this default might be overridden by plain title style
\newcommand\makebeamertitle{\frame{\maketitle}}%
% (ERT) argument for the TOC

\AtBeginDocument{%
  \let\origtableofcontents=\tableofcontents
  \def\tableofcontents{\@ifnextchar[{\origtableofcontents}{\gobbletableofcontents}}
  \def\gobbletableofcontents#1{\origtableofcontents}
}

\usepackage{hyperref} %For link- references


\newcommand{\code}[1]{\textit{#1}} %Format for python, in case I wise for something different change textit


%%%%%%%%%%%%%%%%%%%%%%%%%%%%%% User specified LaTeX commands.
\usepackage{tikz}
\usetikzlibrary{positioning}
\usepackage{appendixnumberbeamer}

\usetheme[progressbar=frametitle,block=fill]{metropolis}

% code
\usepackage{listings} 

% margin
\setbeamersize{text margin right=1.5cm}

% colors
\colorlet{DarkRed}{red!70!black}
\setbeamercolor{normal text}{fg=black}
\setbeamercolor{alerted text}{fg=DarkRed}
\setbeamercolor{progress bar}{fg=DarkRed}
\setbeamercolor{button}{bg=DarkRed}


% width of seperators
\makeatletter
\setlength{\metropolis@titleseparator@linewidth}{1pt}
\setlength{\metropolis@progressonsectionpage@linewidth}{1pt}
\setlength{\metropolis@progressinheadfoot@linewidth}{1pt}
\makeatother

% new alert block
\newlength\origleftmargini
\setlength\origleftmargini\leftmargini
\setbeamertemplate{itemize/enumerate body begin}{\setlength{\leftmargini}{4mm}}
\let\oldalertblock\alertblock
\let\oldendalertblock\endalertblock
\def\alertblock{\begingroup \setbeamertemplate{itemize/enumerate body begin}{\setlength{\leftmargini}{\origleftmargini}} \oldalertblock}
\def\endalertblock{\oldendalertblock \endgroup}
\setbeamertemplate{mini frame}{}
\setbeamertemplate{mini frame in current section}{}
\setbeamertemplate{mini frame in current subsection}{}
\setbeamercolor{section in head/foot}{fg=normal text.bg, bg=structure.fg}
\setbeamercolor{subsection in head/foot}{fg=normal text.bg, bg=structure.fg}

% footer
\makeatletter
\setbeamertemplate{footline}{%
    \begin{beamercolorbox}[colsep=1.5pt]{upper separation line head}
    \end{beamercolorbox}
    \begin{beamercolorbox}{section in head/foot}
      \vskip1pt\insertsectionnavigationhorizontal{\paperwidth}{}{\hskip0pt plus1filll \insertframenumber{} / \inserttotalframenumber \hskip2pt}\vskip3pt% 
    \end{beamercolorbox}%
    \begin{beamercolorbox}[colsep=1.5pt]{lower separation line head}
    \end{beamercolorbox}
}
\makeatother

% toc
\setbeamertemplate{section in toc}{\hspace*{1em}\inserttocsectionnumber.~\inserttocsection\par}
\setbeamertemplate{subsection in toc}{\hspace*{2em}\inserttocsectionnumber.\inserttocsubsectionnumber.~\inserttocsubsection\par}

\makeatother
\title{Seventh exercise class \vspace{-2mm}}
\subtitle{Class 5 \\Introduction to numerical programming and analysis \vspace{-4mm} } 
\author{Asker Nygaard Christensen}
\date{Spring 2021}

\begin{document}


{
\setbeamertemplate{footline}{} 
\begin{frame}

\maketitle

\begin{tikzpicture}[overlay, remember picture]
\node[below left=0cm and 0cm of current page.north east] 
{\includegraphics[width=3 cm]{ku_logo.png}};
\end{tikzpicture}

\end{frame}
}

\addtocounter{framenumber}{-1}

\begin{frame}<beamer>
\frametitle{Plan}

\tableofcontents[]
\end{frame}

\section{Inaugural project}
\begin{frame}{Inaugural project}
\vspace{2cm}
\begin{alertblock}{Inaugural project}
    By now you should all have received feed-back. \\
    I was generally really impressed by your projects. There weren't any glaring recurring mistakes, but I have made a \href{https://github.com/AskerNC/Teaching_notes/blob/master/Inauguralproject2021.ipynb}{\underline{small notebook}}, representing my most recurring comments.
\end{alertblock}
    
\end{frame}
\section{Next assignment -repeated from last time}
\begin{frame}{Data Analysis Project -repeated}

    It might already be a good idea, to start thinking about what you wanna do and what kinda data you wanna use in the next assignment \newline
    You can see the assignment at \href{https://github.com/NumEconCopenhagen/lectures-2021/blob/master/projects/DataProject.pdf}{\underline{Github}}, but basically you have to download data from an online source and do some empirical analysis (key figures and graphs).
    You get to choose the data and what kind of analysis you wanna do yourself. There are LOADS of possibilities, so choose something you find interesting and seems manageable. Recreating a already existing figure using python is also good. \newline
    You can see all the previous projects by searching 'projects-2019' and 'projects-2020' at the \href{https://github.com/NumEconCopenhagen}{\underline{NumEconCopenhagen github account}}. \newline
    Also if you wanna do some regressions you can use \href{https://www.statsmodels.org/stable/user-guide.html}{\underline{statsmodels}}, but regressions are not a prerequisite for doing the assignment, a beautiful figure is just as good. \newline
    Since these projects are idiosyncratic, you'll find less directly applicable code in the lectures and PS. But they are still good for inspiriation, along with googling your problems.
\end{frame}

\begin{frame}{Ideas -repeated}
    \begin{itemize}
        \item Investegating the home-field advantage using Covid-variation. \href{https://understat.readthedocs.io/en/latest/}{\underline{This}} should be able to the job. But you can also download excel sheets from \href{https://www.football-data.co.uk/data.php}{\underline{here}} (Inspiration from \href{http://ftp.iza.org/dp13578.pdf}{\underline{this paper}})
        \item Twitter data (See for example \href{https://www.toptal.com/python/twitter-data-mining-using-python}{\underline{This guide}}
        \item Google trends using \href{https://towardsdatascience.com/google-trends-api-for-python-a84bc25db88f}{\underline{pytrend}}
        \item You recreate graphs from 'Capital in the 21st century' and 'Captial \& Ideology', either from  \href{https://www.quandl.com/data/PIKETTY-Thomas-Piketty}{\underline{QUANDLE}} or \href{http://piketty.pse.ens.fr/en/ideology}{\underline{Piketty's personal website}}
        \item \href{https://wid.world/data/}{\underline{World Inequality Database}}
        \item \href{https://pandas-datareader.readthedocs.io/en/latest/remote_data.html}{\underline{Pandas\_datareader}}, from L08 has the capability to load from multiple sources, including the World Bank, OECD and stock data. 
        \item Denmarks statistics can be accessed using pydst, also L08
        \item There is even a python package for downloading \href{https://imdbpy.github.io/}{\underline{IMDB data}} (A group actually used IMDB data last year, \href{https://github.com/NumEconCopenhagen/projects-2019-credible-threats/tree/master/dataproject}{\underline{Credible threats}}, although they downloaded the data manually)
    \end{itemize}
\end{frame}

\begin{frame}{My tips on Pandas and data science generally -repeated}
    Data science can be an excruciating job. Because you're doing self-chosen projects, you won't be able to rely as much on the lecture and PS. Remember to reach out when you're having problems.\\
    There are loads of python guides and stack-flow answers on the internet, so the right google search can also be your saviour. \\
    You been introduced to many different ways of referencing data. I'd recommend using \code{.loc[I,columns]} mostly in the beginning, as it is the most versatile. Instead of the conditions implicit in \code{I}, you can also use a list of index-numbers. \\
    When creating Boolean condition, I'd also recommend being explicit with your brackets. Also, remember '\code{\&}' is the bit-wise 'and' operator, and '$\vert$' is the bit-wise 'or' operator (Depending on your keyboard is could be AltGr+'the key with $\vert$ on it' (close to backspace), or Alt+i), also called pipe or vertical line.
\end{frame}



\section{Problem set 4, Q2.1-Q2.3}

\begin{frame}{My notes on problem set 3 (also, see lecture 8)}
\textbf{Q 2.1 Import national account data from Denmark Statistics} \\
See section 2.1 in lecture 8 \newline
In step 2 and 4, the '\#'-hint uses \code{nah1\_true}, you only need to write \code{nah1} (because that's the name you give the dataset in step 1). \newline
Remember to look at the data between steps, to make sure your operations are doing what you expect them to do.
\begin{itemize}
    \item[Step 2] For the first line, maybe have a look at the \href{https://pandas.pydata.org/pandas-docs/stable/reference/api/pandas.DataFrame.rename.html}{\underline{documentation}} of \code{.rename()}, and note that you're renaming the columns\\
    \item[Step 3] First notice that \code{[Y, C, G, I, X, M]} is the values of \code{var\_dict}, so you do not need to make a new list \\
    The answers creates the condition for a row to be kept, by looping through the elements and using the '|' (or)-operator. An easier approach is simply to use the \code{isin()}-method: \\ \code{I = nah1.variable.isin(var\_dict.values())}
    \item[Step 4] Run the answer cell again, such that max year is 2020 (all DST Qs)
\end{itemize}
\end{frame}


\begin{frame}{My notes on problem set 3}

\begin{itemize}
    \item[Q 2.2] \textbf{Merge}. See section 2.2 of lecture 8 on merging. Remember that joining is like merging, but on the index.
    \item[Q 2.3] \textbf{Split-apply-combine} can be a bit tricky, so have a look at section 4.2 in the lecture, and don't be afraid to peek at the answer. \\
    Also, it's \code{nah1} not \code{nah1\_final}. \\
    The question mark in \code{h1\_alt[?]}, is just what you want to call the column where your storing your data, the answers calls this \code{index\_transform}.
\end{itemize}
\end{frame}
\section{Q3.1-Q3.2 will be done together}
\section{Q3.3}
\begin{frame}{Q3.3}
    I think it is intentional there that is less help on the last question, so I won't write as explicit hints for this one. \\
    But try to first think of the steps you need to go through: Merging the data sets. Calculating the data you want. Plotting. \\
    The answer performs some of the calculations in the same lines as plotting, but I think it is easier to understand conceptually to separate it. \\
    The variables you need to calculate are: The yearly log-difference within each municipality, and the average yearly log-difference for each municipality. \\
    Also, you need to sort the data by municipality and date, before calculating, to make sure the yearly differences are between adjacent years.
\end{frame}


\end{document}