\documentclass[10pt,danish,t,10pt]{beamer}
\usepackage{lmodern}
\usepackage[T1]{fontenc}
\usepackage[utf8]{inputenc}
\setlength{\parskip}{\smallskipamount}
\setlength{\parindent}{0pt}
\usepackage{babel}
\usepackage{amsmath}
\usepackage{amssymb}
\usepackage{graphicx}
\usepackage{comment} % For longer notes
\ifx\hypersetup\undefined
  \AtBeginDocument{%
    \hypersetup{unicode=true}
  }
\else
  \hypersetup{unicode=true}
\fi
%\usepackage{breakurl}

\usepackage{tikzsymbols}% Smileys and stuf

\makeatletter
%%%%%%%%%%%%%%%%%%%%%%%%%%%%%% Textclass specific LaTeX commands.
% this default might be overridden by plain title style
\newcommand\makebeamertitle{\frame{\maketitle}}%
% (ERT) argument for the TOC

\AtBeginDocument{%
  \let\origtableofcontents=\tableofcontents
  \def\tableofcontents{\@ifnextchar[{\origtableofcontents}{\gobbletableofcontents}}
  \def\gobbletableofcontents#1{\origtableofcontents}
}

\usepackage{hyperref} %For link- references


\newcommand{\code}[1]{\textit{#1}} %Format for python, in case I wise for something different change textit


%%%%%%%%%%%%%%%%%%%%%%%%%%%%%% User specified LaTeX commands.
\usepackage{tikz}
\usetikzlibrary{positioning}
\usepackage{appendixnumberbeamer}

\usetheme[progressbar=frametitle,block=fill]{metropolis}

% code
\usepackage{listings} 

% margin
\setbeamersize{text margin right=1.5cm}

% colors
\colorlet{DarkRed}{red!70!black}
\setbeamercolor{normal text}{fg=black}
\setbeamercolor{alerted text}{fg=DarkRed}
\setbeamercolor{progress bar}{fg=DarkRed}
\setbeamercolor{button}{bg=DarkRed}


% width of seperators
\makeatletter
\setlength{\metropolis@titleseparator@linewidth}{1pt}
\setlength{\metropolis@progressonsectionpage@linewidth}{1pt}
\setlength{\metropolis@progressinheadfoot@linewidth}{1pt}
\makeatother

% new alert block
\newlength\origleftmargini
\setlength\origleftmargini\leftmargini
\setbeamertemplate{itemize/enumerate body begin}{\setlength{\leftmargini}{4mm}}
\let\oldalertblock\alertblock
\let\oldendalertblock\endalertblock
\def\alertblock{\begingroup \setbeamertemplate{itemize/enumerate body begin}{\setlength{\leftmargini}{\origleftmargini}} \oldalertblock}
\def\endalertblock{\oldendalertblock \endgroup}
\setbeamertemplate{mini frame}{}
\setbeamertemplate{mini frame in current section}{}
\setbeamertemplate{mini frame in current subsection}{}
\setbeamercolor{section in head/foot}{fg=normal text.bg, bg=structure.fg}
\setbeamercolor{subsection in head/foot}{fg=normal text.bg, bg=structure.fg}

% footer
\makeatletter
\setbeamertemplate{footline}{%
    \begin{beamercolorbox}[colsep=1.5pt]{upper separation line head}
    \end{beamercolorbox}
    \begin{beamercolorbox}{section in head/foot}
      \vskip1pt\insertsectionnavigationhorizontal{\paperwidth}{}{\hskip0pt plus1filll \insertframenumber{} / \inserttotalframenumber \hskip2pt}\vskip3pt% 
    \end{beamercolorbox}%
    \begin{beamercolorbox}[colsep=1.5pt]{lower separation line head}
    \end{beamercolorbox}
}
\makeatother

% toc
\setbeamertemplate{section in toc}{\hspace*{1em}\inserttocsectionnumber.~\inserttocsection\par}
\setbeamertemplate{subsection in toc}{\hspace*{2em}\inserttocsectionnumber.\inserttocsubsectionnumber.~\inserttocsubsection\par}

\makeatother
\title{Tenth exercise class \vspace{-2mm}}
\subtitle{Class 5 \\Introduction to numerical programming and analysis \vspace{-4mm} } 
\author{Asker Nygaard Christensen}
\date{Spring 2021}

\begin{document}


{
\setbeamertemplate{footline}{} 
\begin{frame}

\maketitle

\begin{tikzpicture}[overlay, remember picture]
\node[below left=0cm and 0cm of current page.north east] 
{\includegraphics[width=3 cm]{ku_logo.png}};
\end{tikzpicture}

\end{frame}
}

\addtocounter{framenumber}{-1}

\begin{frame}<beamer>
\frametitle{Plan}

\tableofcontents[]
\end{frame}


\section{Data project}



\section{Problem set 6}

\begin{frame}{My notes:}
    \begin{alertblock}{General tip}
    When doing \code{scipy.linalg}, Google is your friend. \\
    For any standard matrix-operation you want to do, there will (likely) be a corresponding \code{scipy.linalg}-function.
    
    \end{alertblock}
    \begin{enumerate}
        \item[2.1, A:] You want to find the \underline{det}erminant of the \underline{inv}erse of the dot-product of two matrices. For dot-product you can use \code{np.dot()} but also the \code{@}-operator.
        \item[2.1, B:] It should maybe say using \code{scipy.linalg}. Note that each equation can be solved seperately.
        \item[2.2:] For the \code{gauss\_jordan()}-function to work, you need to add \code{e} as a 4th column in \code{F}. \\ 
        \code{np.column\_stack((F,e))} is the easiest in this case, but there are many others, I'll show you.
    \end{enumerate}

\end{frame}
\begin{frame}{Problem 2.3, Symbolic}
\begin{itemize}
    \item Remember to initiate unknown variables in sympy using \code{sm.symbols()}
    \item And also notice that you need to use the sympy sine-function (\code{sm.sin()})
    \item For the remaining sympy operators you need, I'd use google. Notice that their tutorial/documenation loads sympy as:\\
    \code{from sympy import *}
    \item You can also look through Monday's notebook, which uses all the relevant sympy functions.
\end{itemize}
\end{frame}


\section{P. 3, Solow model and root-finders, 16:15}

\begin{frame}{Problem 3, Solow model}
\begin{itemize}
    \item[A:] Use the answer from above with solving equations symbolically. Notice that the default return of the solver is a list, even when there is only one solution, so you need to extract the first element of this list to get pretty printing.
    \item[B:] \code{sm.lambdify((args),f)} is like \code{lambda args: f}, for symbolic arguments. In this setting you can use your answer to A as \code{f}.
    \item[C:] There are multiple ways of using \code{root\_scalar}. 'Brentq'-method, which requires an bounds (called \code{brackets} for \code{root\_scalar}), 'bisect' is also possible with the same needs. \\
    'newton' method is doesn't need bounds, but does need a first derivative and many more.
    \item[D:] Using CES-production function is (relatively) easy because we're using Python instead of maths
    \end{itemize}
\end{frame}

\begin{frame}{Solving Solow, an alternative }
    An alternative to using \code{root\_scalar}, would be to simply simulate the model, I find this to be the most intuitive way. Inserting a $\tilde{k_{0}}>0$ in the transition equation:
    $$
    \tilde{k}_{t+1}= \frac{1}{(1+n)(1+g)}[sf(\tilde{k}_{t})+(1-\delta)\tilde{k}_{t}]
    $$
    $$
    \tilde{k}_{1}= \frac{1}{(1+n)(1+g)}[sf(\tilde{k}_{0})+(1-\delta)\tilde{k}_{0}]
    $$
    and iterating from that point to find $\tilde{k_{1}} \rightarrow \tilde{k_{2}}\rightarrow \dots$, will eventually lead  (approximately) to the steady state 
    $\tilde{k}^{\ast}$. (where $\tilde{k}_{t}=\tilde{k}_{t+1}$)
\end{frame}
\end{document}